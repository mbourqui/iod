\documentclass[11pt,a4paper,twoside,onecolumn,titlepage]{report}
\usepackage[utf8x]{inputenc}
\usepackage{ucs}
\usepackage{amsmath}
\usepackage{amsfonts}
\usepackage{amssymb}
\usepackage{enumerate}

\renewcommand{\thesection}{\arabic{section}}

\title{Résolution d'un problème d'optimisation différentiable}



\begin{document}

\maketitle

\section*{Énoncé du problème}

Trouver (une approximation de) la solution du problème suivant en appliquant le théorème de la plus forte pente:
\begin{equation}
\min_{x\in\mathbb{R}^2} (x_1-2)^4 + (x_1-2)^2x_2^2 + (x_2+1)^2
\end{equation}

\section*{Réponses aux questions}

\begin{enumerate}[(a)]
\item\label{pfp} Implémenter la méthode de plus forte pente (Algorithme 11.3) à l'aide du logiciel MATLAB. Déterminer la taille du pas en appliquant la recherche linéaire, Algorithme 11.2 (les deux conditions de Wolfe).

%
% Mettre la réponse ici
%

\item\label{pas} Implémenter une fonction qui donne la taille du pas suivant:
\begin{equation}
\alpha_k = \frac{\bigtriangledown f(x_k)^T \bigtriangledown f(x_k)}{\bigtriangledown f(x_k)^T \bigtriangledown^2 f(x_k) \bigtriangledown f(x_k)}
\end{equation}

Quelle est la nature de ce pas? D'où cette formule vient-elle?

%
% Mettre la réponse ici
%

\item Comparer le comportement de l'algorithme en utilisant les pas (\ref{pfp})\ et (\ref{pas}).

%
% Mettre la réponse ici
%

\item Comparer la methode de plus forte pente et la methode quasi-Newton (qui est déjà implementée -- Série 3).

%
% Mettre la réponse ici
%

\end{enumerate}
\end{document}